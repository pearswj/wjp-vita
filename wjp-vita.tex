
\documentclass[oneside,a4paper]{article}
\usepackage{amsmath}
\usepackage{amsfonts}
\usepackage{amssymb}
\usepackage{graphicx}
\usepackage{multicol}
\usepackage{natbib,multirow}
\usepackage[bf,small]{caption2}
\usepackage{parskip}
\usepackage{array}
\usepackage{color}
\usepackage[usenames,dvipsnames]{xcolor}
%\usepackage{fancyhdr}
%\usepackage{hyperref}

\RequirePackage{fancybox}

\setlength{\textwidth}{17.2cm}
\setlength{\textheight}{25.2cm}
\setlength{\topmargin}{-2cm}
\setlength{\oddsidemargin}{-0.6cm}
\setlength{\evensidemargin}{-0.4cm}
\setlength{\headheight}{9pt}

\hbadness=100000
\hfuzz=50pt

\RequirePackage[small]{titlesec}
\titleformat{\section}
{\large\bfseries}{\large\thesection}{1.5em}{}
\titleformat{\subsection}
{\normalsize\bfseries}{\normalsize\thesubsection}{1.5em}{\textit}
\titleformat{\subsubsection}
{\small\bfseries}{\small\thesubsubsection}{1.5em}{\textit}

\makeatletter
\newenvironment{tablehere}
  {\def\@captype{table}}
  {}

\newenvironment{figurehere}
  {\def\@captype{figure}}
  {}
\makeatother

\renewcommand\figurename{Fig.}
\renewcommand{\captionlabeldelim}{}

%\pagestyle{fancy}
%\setlength{\headheight}{15.2pt}
%\renewcommand{\headrulewidth}{0pt}
%\lfoot{\small\textcolor{Gray}{\it \VCRevision}}
%\rfoot{\small\textcolor{Gray}{\it \VCAuthor}}


%-------------------------------------------

\begin{document}
\immediate\write18{sh ./vc}
\input{vc}

\noindent{\Huge \textbf{Will Pearson}}\\

\noindent \textit{wpearson01@gmail.com\hspace{5mm}$\cdot$\hspace{5mm}07882161421\hspace{5mm}$\cdot$\hspace{5mm}Badger House, Oriel Gardens, Bath. BA1 7AS}

\small
\vspace{0.5cm}
%\vspace{1.0cm}

I am a student in my fourth year of studying for an \textbf{MEng in Structural Engineering with Architectural Studies} at Sheffield University. I have a great interest in the use and application of computing in the areas of structural engineering and architecture and am constantly looking to develop my knowledge of software and programming alongside my regular studies.

Last year I undertook my dissertation module on the subject of Structural Layout Optimization; I investigated the scope and limitations of a program written by my dissertation tutor. This involved learning some C$++$ in order to modify the program and also gave me the opportunity to use the University's High Performance Computer Cluster (Iceberg).

\section*{Degree Modules}

Over the last three years of my MEng degree my studies have included the following:

\begin{itemize}
\item structural mechanics and analysis (elastic and plastic, static and dynamic)
\item engineering mathematics
\item geotechnics
\item engineering design, including drawing and surveying
\item structural design projects (group and individual)
\item bridge engineering
\item architecture modules (lectures and studio)
\end{itemize}

My final year modules include F.E. modelling, fracture mechanics and further design projects.

\section*{Education}

\hspace{-6pt}\begin{tabular}{>{\it}lll}
%University of Sheffield & \hspace{-1mm}$\cdot$\hspace{3mm}2008 - 2012 & \hspace{-1mm}$\cdot$\hspace{3mm}2.1 (at end of 3\textsuperscript{rd} year) \\
%Beechen Cliff Sixth Form, Bath & \hspace{-1mm}$\cdot$\hspace{3mm}2006 - 2008 & \hspace{-1mm}$\cdot$\hspace{3mm}A Level: Maths (A); Physics (B); Art (A). AS level: Computing (A) \\
%Beechen Cliff School, Bath & \hspace{-1mm}$\cdot$\hspace{3mm}2001 - 2006 & \hspace{-1mm}$\cdot$\hspace{3mm}GCSE: 11 A--C grades
2008--present & University of Sheffield & Working towards a 2:1. (Due to graduate in 2012.) \\
2006--2008 & Beechen Cliff Sixth Form, Bath & A Level: Maths (A); Physics (B); Art (A). AS level: Computing (A) \\
2001--2006 & Beechen Cliff School, Bath & GCSE: 11 A--C grades
\end{tabular}

\section*{Employment}

\subsection*{Internship {\it -- January 2012 to present}}
{\it LimitState Ltd., Sheffield}

I am currently working for LimitState Ltd.\ as part of the Sheffield Internship Scheme. During the course of this internship I will be involved in a number of projects including the testing and development of civil engineering software and the writing of documentation.

\subsection*{Research Assistant {\it -- July to October 2011}}
{\it Computational Mechanics and Design (CMD) Research Group, Department of Civil and Structural Engineering, University of Sheffield}

This position included formatting research papers for submission (using the \LaTeX~typesetting language) as well as drawing figures and graphs (using the Asymptote language) to be included within. I also had the chance to get involved with some of the research happening in the department including building and processing data from Finite Element models (using LS-DYNA/LS-PrePost) and writing documentation in a wiki format.

During this time my ability to adapt quickly to these new software tools was thoroughly tested. I also learnt how to manage my time with multiple on-going projects and how team members, such as those in the CMD group, work collaboratively using version control systems such as Subversion. The skills I developed in formatting have since been transferred back to my studies when I took responsibility for the compilation and presentation of a feasibility report for a recent group project; My group went on to ``win'' the bid.

\subsection*{Assistant (part-time) {\it -- November 2006 to September 2008, July to September 2009}}
{\it W. M. Morrisons, Bath}

\section*{Other Industry Experience} % Other Industry Experience/Work Experience?

\hspace{-6pt}\begin{tabular}{>{\it}ll}
%Summer 2011 & Research Assistant (Dept. of Civil \& Struct. Eng., University of Sheffield) \\ % remove?
19th August 2010 & Bath Riverside Development meetings (Buro Happold, Bath office) \\
18th August 2010	 & London 2012 Team Stadium design office inc. site visit (Buro Happold) \\
May 2008	 & Site visit to St. David's 2 Development, Cardiff (Whitbybird) \\
June 2007 & 3 day intro placement (Buro Happold, Bath Office)
\end{tabular}

My work experience and site visits have been very valuable in helping with the important decisions concerning my education as a structural engineer.

\section*{Key Achievements}

\begin{itemize}
\item ``Winning'' bid in several group design/feasibility projects
\item Duke of Edinburgh Silver Award
\item Duke of Edinburgh Gold Award (Expedition only)
\end{itemize}

\section*{Key Skills}

\subsection*{Technical Skills}
\begin{multicols}{3}

\subsubsection*{Software}
\begin{itemize}
\item AutoCAD
\item Ansys (APDL)
\item LimitState
\end{itemize}

\subsubsection*{Formatting}
\begin{itemize}
\item \LaTeX
\item Asymptote
\item Adobe Creative Suite
\end{itemize}

\subsubsection*{Programming}
\begin{itemize}
\item MATLAB
\item CLI (Linux/Mac OS X)
\item Subversion/Git
\end{itemize}

\end{multicols}

I am also in the process of learning C$++$ and have some experience of compiling and debugging C/C$++$ on Linux (CLI) and Mac OS X (Xcode). I feel that I have an {\bf aptitude} for learning new software packages and programming languages. I am constantly developing my knowledge on an ad-hoc basis, for example  through the use of software under development from the community repositories in Linux.

\subsection*{Personal Skills and Interests}

\hspace{-6pt}\begin{tabular}{>{\it}lp{430pt}}
Practical & I constantly challenge myself with personal projects. I have built and maintained my own bikes and, more recently, refurbished an old PC on which to learn Linux. I have enjoyed researching the required skills to complete these tasks. \\[4pt]
Active & I play ice hockey for the University of Sheffield, a sport which demands a high level of team work and physical commitment on and off the ice. This year I have been involved in the running of the club in both the executive committee and the coaching committee. \\[4pt]
Independent & In 2008 I completed the expedition part of my Duke of Edinburgh Gold Award -- a four day hike through the Massif Central of France, starting in the ski resort of Le Mont-Dore and finishing at Volvic. \\[4pt]
& I have a full UK driving licence, with no endorsements.
\end{tabular}

\section*{References}

References are available on request.
\\\\\\
\small\texttt{\textcolor{Gray}{\VCDateTEX\;\;(\GITAbrHash)}}

\end{document}

